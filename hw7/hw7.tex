% -----------------------------*- LaTeX -*------------------------------
\documentclass[UTF8]{report}
% ------------------------------------------------------------------------
% Packages
% ------------------------------------------------------------------------
\usepackage{ctex} % 支持中文
\usepackage[body={7in, 9in},left=1in,right=1in]{geometry} % 改变页边距
\usepackage{amsmath} % AMS 的数学宏包
\usepackage{amsfonts} % AMS 的数学字体宏包
\usepackage{amssymb} % AMS 符号库
\usepackage{bm} % 数学公式中的黑斜体h
\usepackage{amsthm} % AMS 的定理环境宏包
\usepackage{graphicx} % 插图
\usepackage{subfigure} % 插子图
\usepackage{nicefrac} % 好看的分数
\usepackage{mathrsfs} % mathscr font
\usepackage{caption} % caption
\usepackage{algorithm,algorithmicx} % 伪代码支持宏包
\usepackage[noend]{algpseudocode} % 伪代码
\usepackage{fancyhdr} % 设置页眉、页脚
\usepackage{adjustbox} % 图片尺寸自动调整
\usepackage{esint} % 积分符号
\usepackage{mathtools} % 数学宏包的重要补充
\usepackage{upgreek} % 数学环境的直立希腊字母
\usepackage{enumitem} % 使用enumitem宏包, 改变列表项的格式
\usepackage{color} % 支持彩色
\usepackage{extarrows} % 任意长度的箭头
\usepackage{tikz} % 绘图
\usepackage{forest} % 绘树
\usepackage{xcolor} % 颜色宏包
\usepackage{breqn} % 公式自动换行
\usepackage{fontsize} % 字体大小
\usepackage[framemethod=TikZ]{mdframed} % 给文字加框
\usepackage{fontspec} % 字体库
\usepackage{bigstrut} % 用于表格中的换行
\usepackage{multirow} % 表格中多行单元格合并
\usepackage{multicol} % 表格中多列单元格合并
\usepackage{longtable} % 长表格
\usepackage{rotating} % 旋转图形和表格      以上三者用于绘制三线表
\usepackage{booktabs} % 三线表宏包
\usepackage{forest} % 绘制语法树
\usepackage{scribe} % Scribe 模板
\usepackage{diagbox} % 表格斜线
\usepackage{listings} % 插入代码
\usepackage{verbatim} % 多行注释
\usepackage{ifplatform} % 检测编译平台
\usepackage{hyperref} % 超链接
\usepackage{mathrsfs} % 花体
\usepackage{pgffor} % foreach
\usepackage{circuitikz} % 画电路图
\usepackage{svg} % 插入svg
\usetikzlibrary{shapes.geometric, arrows} % 引入流程图需要的库
\usetikzlibrary{automata} % 引入automata库
\usetikzlibrary{shapes,arrows,positioning,chains} % 引入positioning库
% ------------------------------------------------------------------------
% Macros
% ------------------------------------------------------------------------
%~~~~~~~~~~~~~~~
% Utility latin
%~~~~~~~~~~~~~~~
\newcommand{\ie}{\textit{i.e.}}
\newcommand{\eg}{\textit{e.g.}}
%~~~~~~~~~~~~~~~
% Environment shortcuts
%~~~~~~~~~~~~~~~
\newcommand{\balign}[1]{\ealign{\begin{align}#1\end{align}}}
\newcommand{\baligns}[1]{\ealigns{\begin{align*}#1\end{align*}}}
\newcommand{\bitemize}[1]{\eitemize{\begin{itemize}#1\end{itemize}}}
\newcommand{\benumerate}[1]{\eenumerate{\begin{enumerate}#1\end{enumerate}}}
%~~~~~~~~~~~~~~~
% Text with quads around it
%~~~~~~~~~~~~~~~
\newcommand{\qtext}[1]{\quad\text{#1}\quad}
%~~~~~~~~~~~~~~~
% Shorthand for math formatting
%~~~~~~~~~~~~~~~
\newcommand{\mbb}[1]{\mathbb{#1}}
\newcommand{\mbi}[1]{\boldsymbol{#1}} % Bold and italic (math bold italic)
\newcommand{\mbf}[1]{\mathbf{#1}}
\newcommand{\mc}[1]{\mathcal{#1}}
\newcommand{\mrm}[1]{\mathrm{#1}}
\newcommand{\tbf}[1]{\textbf{#1}}
\newcommand{\tsc}[1]{\textsc{#1}}
%\def\\langle {{\langle }}
%\def\\rangle {{\rangle }}
\newcommand{\sT}{\sf T}
\newcommand{\grad}{\nabla}
\newcommand{\Proj}{\Pi}
%~~~~~~~~~~~~~~~
% Common sets 定义数集符号
%~~~~~~~~~~~~~~~
\newcommand{\R}{\mathbb{R}}
\newcommand{\Z}{\mathbb{Z}}
\newcommand{\Q}{\mathbb{Q}}
\newcommand{\N}{\mathbb{N}}
\newcommand{\C}{\mathbb{C}}
\newcommand{\reals}{\mathbb{R}} % Real number symbol
\newcommand{\integers}{\mathbb{Z}} % Integer symbol
\newcommand{\rationals}{\mathbb{Q}} % Rational numbers
\newcommand{\naturals}{\mathbb{N}} % Natural numbers
\newcommand{\complex}{\mathbb{C}} % Complex numbers
%~~~~~~~~~~~~~~~
% Common functions
%~~~~~~~~~~~~~~~
\renewcommand{\exp}[1]{\operatorname{exp}\left(#1\right)} % Exponential
\newcommand{\indic}[1]{\mbb{I}\left(#1\right)} % Indicator function
\newcommand{\indicsub}[2]{\mbb{I}_{#2}\left(#1\right)} % Indicator function
\newcommand{\argmax}{\mathop\mathrm{arg\, max}} % Defining math symbols
\newcommand{\argmin}{\mathop\mathrm{arg\, min}}
\renewcommand{\arccos}{\mathop\mathrm{arccos}}
\newcommand{\dom}{\mathop\mathrm{dom}} % Domain
\newcommand{\range}{\mathop\mathrm{range}} % Range
\newcommand{\diag}{\mathop\mathrm{diag}}
\newcommand{\tr}{\mathop\mathrm{tr}}
\newcommand{\abs}{\mathop\mathrm{abs}}
\newcommand{\card}{\mathop\mathrm{card}}
\newcommand{\sign}{\mathop\mathrm{sign}}
\newcommand{\prox}{\mathrm{prox}} % prox
\newcommand{\rank}[1]{\mathrm{rank}(#1)}
\newcommand{\supp}[1]{\mathrm{supp}(#1)}
\newcommand{\norm}[1]{\lVert#1\rVert}
%~~~~~~~~~~~~~~~
% Common probability symbols
%~~~~~~~~~~~~~~~
\newcommand{\family}{\mathcal{P}} % probability family / statistical model
\newcommand{\iid}{\stackrel{\mathrm{iid}}{\sim}}
\newcommand{\ind}{\stackrel{\mathrm{ind}}{\sim}}
\newcommand{\E}{\mathbb{E}} % Expectation symbol
\newcommand{\Earg}[1]{\E\left[#1\right]}
\newcommand{\Esubarg}[2]{\E_{#1}\left[#2\right]}
\renewcommand{\P}{\mathbb{P}} % Probability symbol
\newcommand{\Parg}[1]{\P\left(#1\right)}
\newcommand{\Psubarg}[2]{\P_{#1}\left[#2\right]}
%\newcommand{\Cov}{\mrm{Cov}} % Covariance symbol
%\newcommand{\Covarg}[1]{\Cov\left[#1\right]}
%\newcommand{\Covsubarg}[2]{\Cov_{#1}\left[#2\right]}
%\newcommand{\model}{\mathcal{P}} % probability family / statistical model
%~~~~~~~~~~~~~~~
% Distributions
%~~~~~~~~~~~~~~~
%\newcommand{\Gsn}{\mathcal{N}}
%\newcommand{\Ber}{\textnormal{Ber}}
%\newcommand{\Bin}{\textnormal{Bin}}
%\newcommand{\Unif}{\textnormal{Unif}}
%\newcommand{\Mult}{\textnormal{Mult}}
%\newcommand{\NegMult}{\textnormal{NegMult}}
%\newcommand{\Dir}{\textnormal{Dir}}
%\newcommand{\Bet}{\textnormal{Beta}}
%\newcommand{\Gam}{\textnormal{Gamma}}
%\newcommand{\Poi}{\textnormal{Poi}}
%\newcommand{\HypGeo}{\textnormal{HypGeo}}
%\newcommand{\GEM}{\textnormal{GEM}}
%\newcommand{\BP}{\textnormal{BP}}
%\newcommand{\DP}{\textnormal{DP}}
%\newcommand{\BeP}{\textnormal{BeP}}
%\newcommand{\Exp}{\textnormal{Exp}}
%~~~~~~~~~~~~~~~
% Theorem-like environments
%~~~~~~~~~~~~~~~
%\theoremstyle{definition}
%\newtheorem{definition}{Definition}
%\newtheorem{example}{Example}
%\newtheorem{problem}{Problem}
%\newtheorem{lemma}{Lemma}
%~~~~~~~~~~~~~~~
% 组合数学的模板和作业里用到的一些宏包和自定义命令
%~~~~~~~~~~~~~~~
\renewcommand{\emph}[1]{\begin{kaishu}#1\end{kaishu}}
\newcommand{\falfac}[1]{^{\underline{#1}}}
\newcommand{\binomfrac}[2]{\frac{#1^{\underline{#2}}}{#2!}}
\newcommand{\ceil}[1]{\left\lceil #1 \right\rceil}
\newcommand{\floor}[1]{\left\lfloor #1 \right\rfloor}
\newcommand{\suminfty}[2]{\sum_{#1=#2}^{\infty}}
\newcommand{\suminftyk}[0]{\sum_{k=0}^{\infty}}
\newcommand{\sumint}[3]{\sum_{#1=#2}^{#3}}
\newcommand{\sumintk}[2]{\sum_{k=#1}^{#2}}
\newcommand{\suminti}[2]{\sum_{i=#1}^{#2}}
%~~~~~~~~~~~~~~~
% 定义新命令
%~~~~~~~~~~~~~~~
\newcommand*{\unit}[1]{\mathop{}\!\mathrm{#1}}
\newcommand*{\dif}{\mathop{}\!\mathrm{d}}%微分算子 d
\newcommand*{\pdif}{\mathop{}\!\partial}%偏微分算子
\newcommand*{\cdif}{\mathop{}\!\nabla}%协变导数、nabla 算子
\newcommand*{\laplace}{\mathop{}\!\Delta}%laplace 算子
\newcommand*{\deri}[1]{\mathrm{d} #1}
\newcommand*{\deriv}[2]{\frac{\mathrm{d} #1}{\mathrm{d} {#2}}}
\newcommand*{\derivh}[3]{\frac{\mathrm{d}^{#1} #2}{\mathrm{d} {#3^{#1}}}}
\newcommand*{\pderiv}[2]{\frac{\partial #1}{\partial {#2}}}
\newcommand*{\pderivh}[3]{\frac{\partial^{#1} #2}{\partial {#3^{#1}}}}
\newcommand*{\dderiv}[2]{\dfrac{\mathrm{d} #1}{\mathrm{d} {#2}}}
\newcommand*{\dderivh}[3]{\dfrac{\mathrm{d}^{#1} #2}{\mathrm{d} {#3^{#1}}}}
\newcommand*{\dpderiv}[2]{\dfrac{\partial #1}{\partial {#2}}}
\newcommand*{\dpderivh}[3]{\dfrac{\partial^{#1} #2}{\partial {#3^{#1}}}}
\newcommand{\me}[1]{\mathrm{e}^{#1}}%e 指数
\newcommand{\mi}{\mathrm{i}}%虚数单位
%\newcommand{\mc}{\mathrm{c}}%光速 定义与mathcal冲突
\newcommand{\red}[1]{\textcolor{red}{#1}}
\newcommand{\blue}[1]{\textcolor{blue}{#1}}
%\newcommand{\Rome}[1]{\setcounter{rome}{#1}\Roman{rome}}
%~~~~~~~~~~~~~~~
% 公式环境中箭头符号的简写
%~~~~~~~~~~~~~~~
\newcommand{\ra}{\rightarrow}
\newcommand{\Ra}{\Rightarrow}
\newcommand{\la}{\leftarrow}
\newcommand{\La}{\Leftarrow}
\newcommand{\lra}{\leftrightarrow}
\newcommand{\Lra}{\Leftrightarrow}
\newcommand{\lgla}{\longleftarrow}
\newcommand{\Lgla}{\Longleftarrow}
\newcommand{\lgra}{\longrightarrow}
\newcommand{\Lgra}{\Longrightarrow}
\newcommand{\lglra}{\longleftrightarrow}
\newcommand{\Lglra}{\Longleftrightarrow}
%~~~~~~~~~~~~~~~
% 一些数学的环境设置
%~~~~~~~~~~~~~~~
%\newcounter{counter_exm}\setcounter{counter_exm}{1}
%\newcounter{counter_prb}\setcounter{counter_prb}{1}
%\newcounter{counter_thm}\setcounter{counter_thm}{1}
%\newcounter{counter_lma}\setcounter{counter_lma}{1}
%\newcounter{counter_dft}\setcounter{counter_dft}{1}
%\newcounter{counter_clm}\setcounter{counter_clm}{1}
%\newcounter{counter_cly}\setcounter{counter_cly}{1}
\newtheorem{theorem}{{\hskip 1.7em \bf 定理}}
\newtheorem{lemma}[theorem]{\hskip 1.7em 引理}
\newtheorem{proposition}[theorem]{\hskip 1.7em 命题}
\newtheorem{claim}[theorem]{\hskip 1.7em 断言}
\newtheorem{corollary}[theorem]{\hskip 1.7em 推论}
% \newcommand{\problem}[1]{{\setlength{\parskip}{10pt}\noindent \bf{#1}}}
\newenvironment{solution}{{\noindent \bf 解 \quad}}{}
\newenvironment{remark}{{\noindent \bf 注 \quad}}{}
\newenvironment{definition}{{\noindent \bf 定义 \quad}}{}
\renewenvironment{proof}{{\setlength{\parskip}{7pt}\noindent\hskip 2em \bf 证明 \quad}}{\hfill$\qed$\par}
\newenvironment{example}{{\noindent\bf 例 \quad}}{\hfill$\qed$\par}
%\newenvironment{concept}[1]{{\bf #1\quad} \begin{kaishu}} {\end{kaishu}\par}
%~~~~~~~~~~~~~~~
% 本.tex文档中特殊定义命令
%~~~~~~~~~~~~~~~
\newcommand{\lno}[1]{\overline{#1}}
\newcommand{\NP}{\mathrm{NP}}
\newcommand{\coNP}{\mathrm{coNP}}
% \newcommand{\ISO}{\mathrm{ISO}}
\newcommand{\SAT}{\mathrm{SAT}}
\newcommand{\USAT}{\mathrm{USAT}}
% \newcommand{\threeSAT}{\mathrm{3\text{-}SAT}}
\renewcommand{\P}{\mathrm{P}}
% \mathchardef\mhyphen="2D
% \newcommand{\CNF}{\mathrm{CNF}}
% \newcommand{\DNF}{\mathrm{DNF}}
% \newcommand{\SetSp}{\mathrm{SET\text{-}SPLITTING}}
% \newcommand{\PUZZLE}{\mathrm{PUZZLE}}
% \newcommand{\SPATH}{\mathrm{SPATH}}
% \newcommand{\LPATH}{\mathrm{LPATH}}
% \newcommand{\UHAMPATH}{\mathrm{UHAMPATH}}
\newcommand{\SPACE}{\mathrm{SPACE}}
\newcommand{\NSPACE}{\mathrm{NSPACE}}
\newcommand{\PSPACE}{\mathrm{PSPACE}}
\newcommand{\NPSPACE}{\mathrm{NPSPACE}}
\newcommand{\DFA}{\mathrm{DFA}}
\newcommand{\NFA}{\mathrm{NFA}}
\newcommand{\TQBF}{\mathrm{TQBF}}
% \newcommand{\L}{\mathrm{L}}
\renewcommand{\O}{\mathrm{O}}
\newcommand{\NL}{\mathrm{NL}}
\newcommand{\coNL}{\mathrm{coNL}}
\newcommand{\LADDER}{\mathrm{LADDER_{DFA}}}
\newcommand{\hd}{\mathrm{\text{-}hard}}
\newcommand{\ADD}{\mathrm{ADD}}
\newcommand{\STCN}{\mathrm{STRONGLY\text{-}CONNECTED}}
\newcommand{\PATH}{\mathrm{PATH}}
\newcommand{\A}{\mathrm{A}}
%使用align环境公式换页
\allowdisplaybreaks[4]

\definecolor{dkgreen}{rgb}{0,0.6,0}
\definecolor{gray}{rgb}{0.5,0.5,0.5}
\definecolor{mauve}{rgb}{0.58,0,0.82}
\lstset{
  frame=tb,
  aboveskip=3mm,
  belowskip=3mm,
  showstringspaces=false,
  columns=flexible,
  framerule=1pt,
  rulecolor=\color{gray!35},
  backgroundcolor=\color{gray!5},
  basicstyle={\small\ttfamily},
  numbers=left,
  numberstyle=\ttfamily\color{gray},
  keywordstyle=\color{blue},
  commentstyle=\color{dkgreen},
  stringstyle=\color{mauve},
  breaklines=true,
  breakatwhitespace=true,
  tabsize=3,
}

\lstdefinelanguage{LoongArch}{
  morekeywords={la, lw, addi, sw, li, syscall, beqz, add, move, bge, blt, b, sub, ret, beq, bne},  
  literate={ll.w}{{{\color{blue}ll.w}}}1
           {sc.w}{{{\color{blue}sc.w}}}1
           {addi.d}{{{\color{blue}addi.d}}}1
           {st.d}{{{\color{blue}st.d}}}1
           {st.w}{{{\color{blue}st.w}}}1
           {ldptr.w}{{{\color{blue}ldptr.w}}}1
           {slli.d}{{{\color{blue}slli.d}}}1
           {ld.d}{{{\color{blue}ld.d}}}1
           {stptr.w}{{{\color{blue}stptr.w}}}1
           {slli.w}{{{\color{blue}slli.w}}}1
           {ld.w}{{{\color{blue}ld.w}}}1
           {addi.w}{{{\color{blue}addi.w}}}1
           {add.d}{{{\color{blue}add.d}}}1
           {sub.w}{{{\color{blue}sub.w}}}1
           {li.w}{{{\color{blue}li.w}}}1
           {bstrpick.d}{{{\color{blue}bstrpick.d}}}1
           {alsl.d}{{{\color{blue}alsl.d}}}1,
  morecomment=[l]{\#},
  frame=tb,
  aboveskip=3mm,
  belowskip=3mm,
  showstringspaces=false,
  columns=flexible,
  framerule=1pt,
  rulecolor=\color{gray!35},
  backgroundcolor=\color{gray!5},
  basicstyle={\small\ttfamily},
  numbers=left,
  numberstyle=\ttfamily\color{gray},
  keywordstyle=\color{blue},
  commentstyle=\color{dkgreen},
  stringstyle=\color{mauve},
  breaklines=true,
  breakatwhitespace=true,
  tabsize=3,
}

% 设置超链接样式
\hypersetup{
    colorlinks=true,       % 将链接颜色设置为 true
    linkcolor=magenta,        % 内部链接颜色
    filecolor=magenta,     % 文件链接颜色
    urlcolor=blue,         % URL 链接颜色
    citecolor=green,       % 引用链接颜色
}

\tikzstyle{startstop} = [rectangle, rounded corners, minimum width=3cm, minimum height=1cm,text centered, draw=black, fill=red!30]
\tikzstyle{process} = [rectangle, minimum width=3cm, minimum height=1cm, text centered, draw=black, fill=orange!30]
\tikzstyle{decision} = [diamond, minimum width=3cm, minimum height=1cm, text centered, draw=black, fill=green!30]
\tikzstyle{arrow} = [thick,->,>=stealth]

\ifwindows
    \setmainfont{Times New Roman}
    \setsansfont{Times New Roman}
    \setmonofont{Consolas}
    \setCJKmainfont{SimHei}
    \setCJKsansfont{SimSun}
    \setCJKmonofont{FangSong}
\else
    \setmainfont{Times New Roman}
    \setsansfont{Times New Roman}
    \setmonofont{Menlo}
    \setCJKmainfont{Songti SC}
    \setCJKsansfont{STSong}
    \setCJKmonofont{STFangsong}
\fi

\punctstyle{kaiming}

\begin{document}

\pagestyle{fancy}

\reporttype{Homework}                 % required
\course{Compiler Principle} 				% optional
\coursetitle{语法制导翻译}	    % optional
\semester{Spring 2025}			    % optional
\lecturer{Feng Xiaobing}			% optional
\scribe{2022K8009929010 Zhang Jiawei}			% required
\lecturenumber{7}				% required (must be a number)
\lecturedate{April 17}			% required (omit year)
\maketitle

\noindent
\tbf{7.1}

\begin{enumerate}[label=(\arabic*)]
    \item 消除左递归之后的文法如下所示:
    
    \begin{align*}
        S &\to E \, n \\
        E &\to T \, E' \\
        E' &\to + \, T \, E' \, | \, \varepsilon \\
        T &\to F \, T' \\
        T' &\to * \, F \, T' \, | \, \varepsilon \\
        F &\to ( \, E \, ) \, | \, digit
    \end{align*}

    \item 所得SDD如下表所示:
    
    \begin{table}[H]
        \centering
        \begin{tabular}{|c|c|}
            \hline
            \textbf{产生式} & \textbf{语义规则} \\
            \hline
            $S \to E \, n$ & $S.val = E.val$ \\
            \hline
            $E \to T \, E'$ & $E'.inh = T.val$ \\
            & $E.val = E'.syn$ \\
            \hline
            $E' \to + \, T \, E'_1$ & $E'_1.inh = E'.inh + T.val$ \\
            & $E'.syn = E'_1.syn$ \\
            \hline
            $E' \to \varepsilon$ & $E'.syn = E'.inh$ \\
            \hline
            $T \to F \, T'$ & $T'.inh = F.val$ \\
            & $T.val = T'.syn$ \\
            \hline
            $T' \to * \, F \, T'_1$ & $T'_1.inh = T'.inh \times F.val$ \\
            & $T'.syn = T'_1.syn$ \\
            \hline
            $T' \to \varepsilon$ & $T'.syn = T'.inh$ \\
            \hline
            $F \to ( \, E \, )$ & $F.val = E.val$ \\
            \hline
            $F \to digit$ & $F.val = digit.lexval$ \\
            \hline
        \end{tabular}
    \end{table}

    \item 使用上面得到的SDD,给出表达式 3 * (4 + 5) n 的注释语法分析树:
    
    \begin{figure}[H]
        \centering
        \begin{forest}
            for tree={
                align=center,
                anchor=north,
                inner sep=0.05cm,
                outer sep=0.05cm,
                l sep+=20pt,
                s sep+=10pt,
            }
            [\text{S.val = 27}
                [\text{E.val = 27}
                    [\text{T.val = 27}
                        [\text{F.val = 3}
                            [\text{digit.lexval = 3}]
                        ]
                        [\text{T'.inh = 3} \\ \text{T'.syn = 27}
                            [\text{*}]
                            [\text{F.val = 9}
                                [\text{(}]
                                [\text{E.val = 9}
                                    [\text{T.val = 4}
                                        [\text{F.val = 4}
                                            [\text{digit.lexval = 4}]
                                        ]
                                        [\text{T'.inh = 4} \\ \text{T'.syn = 4}
                                            [\text{$\varepsilon$}]
                                        ]
                                    ]
                                    [\text{E'.inh = 4} \\ \text{E'.syn = 9}
                                        [\text{+}]
                                        [\text{T.val = 5}
                                            [\text{F.val = 5}
                                                [\text{digit.lexval = 5}]
                                            ]
                                            [\text{T'.inh = 5} \\ \text{T'.syn = 5}
                                                [\text{$\varepsilon$}]
                                            ]
                                        ]
                                        [\text{E'.inh = 9} \\ \text{E'.syn = 9}
                                            [\text{$\varepsilon$}]
                                        ]
                                    ]
                                ]
                                [\text{)}]
                            ]
                            [\text{T'.inh = 27} \\ \text{T'.syn = 27}
                                [\text{$\varepsilon$}]
                            ]
                        ]
                    ]
                    [\text{E'.inh = 27} \\ \text{E'.syn = 27}
                        [\text{$\varepsilon$}]
                    ]
                ]
                [\text{n}]
            ]
        \end{forest}
    \end{figure}
\end{enumerate}

\noindent
\tbf{7.2}

\begin{enumerate}[label=(\arabic*)]
    \item B.i = A.i; A.s = B.i + C.s
    \begin{enumerate}
        \item A的综合属性由其子节点定义,符合S属性要求;
        \item B的继承属性由其父节点定义,符合L属性要求;
        \item 没有循环依赖,故存在一致的求值过程。
    \end{enumerate}
    \item B.i = A.i; A.s = B.i + C.s; D.i = A.i + B.s
    \begin{enumerate}
        \item A的综合属性由其子节点定义,符合S属性要求;
        \item B的继承属性由其父节点定义,D的继承属性由其父节点和前方兄弟节点定义,符合L属性要求;
        \item 没有循环依赖,故存在一致的求值过程。
    \end{enumerate}
    \item A.s = B.s + C.s
    \begin{enumerate}
        \item A的综合属性由其子节点定义,符合S属性要求;
        \item 无继承属性语义规则,符合L属性要求;
        \item 没有循环依赖,故存在一致的求值过程。
    \end{enumerate}
\end{enumerate}

\noindent
\tbf{7.3}

仍然先消除左递归:

\begin{align*}
    E &\to T \, E' \\
    E' &\to + \, T \, E' \, | \, \varepsilon \\
    T &\to F \, T' \\
    T' &\to * \, F \, T' \, | \, \varepsilon \\
    F &\to ( \, E \, ) \, | \, num \, | \, var
\end{align*}

设计SDD如下表所示:

\begin{table}[H]
    \centering
    \begin{tabular}{|c|c|}
        \hline
        \textbf{产生式} & \textbf{语义规则} \\
        \hline
        $E \to T \, E'$ & $E'.inh = T.isconst$ \\
        & $E.isconst = E'.syn$ \\
        \hline
        $E' \to + \, T \, E'_1$ & $E'_1.inh = E'.inh + T.isconst$ \\
        & $E'.syn = E'_1.syn$ \\
        \hline
        $E' \to \varepsilon$ & $E'.syn = E'.inh$ \\
        \hline
        $T \to F \, T'$ & $T'.inh = F.isconst$ \\
        & $T.isconst = T'.syn$ \\
        \hline
        $T' \to * \, F \, T'_1$ & $T'_1.inh = T'.inh \times F.isconst$ \\
        & $T'.syn = T'_1.syn$ \\
        \hline
        $T' \to \varepsilon$ & $T'.syn = T'.inh$ \\
        \hline
        $F \to ( \, E \, )$ & $F.isconst = E.isconst$ \\
        \hline
        $F \to num$ & $F.isconst = true$ \\
        \hline
        $F \to var$ & $F.isconst = false$ \\
        \hline
    \end{tabular}
\end{table}

\noindent
\tbf{7.4}

先消除文法左递归:

\begin{align*}
    B &\to 1 \, B' \\
    B' &\to 0 \, B' \, | \, 1 \, B' \, | \, \varepsilon
\end{align*}

然后添加语义动作,写出SDT:

\begin{align*}
    B \to &1 \, B' \quad \{B'.inh = 1 \,,\ B.val = B'.syn\} \\
    B' \to &0 \, B'_1 \quad \{B'_1.inh = B'.inh \times 2 \,,\ B'.syn = B'_1.syn\} \\
    | \, &1 \, B'_1 \quad \{B'_1.inh = B'.inh \times 2 + 1 \,,\ B'.syn = B'_1.syn\} \\
    | \, &\varepsilon \quad \,\,\,\,\,\,\,\, \{B'.syn = B'.inh\}
\end{align*}

\noindent
\tbf{7.5}

\begin{enumerate}[label=(\arabic*)]
    \item 先写出SDD如下:
    
    \begin{align*}
        S \to \text{if} \, (C) \, S_1 \, \text{else} \, S_2 \qquad &L_1 = new(); \\
        &C.false = L_1; \\
        &S_1.next = S.next; \\
        &S_2.next = S.next; \\
        &S.code = C.code \, \| \, S_1.code \, \| \, \text{label} \| \, L_1 \, \| \, S_2.code
    \end{align*}

    对应SDT如下:

    \begin{align*}
        S \to \text{if} \, ( \qquad &\{L_1 = new() \,,\ C.false = L_1\} \\
        C \, ) \qquad &\{S_1.next = S.next\} \\
        S_1 \, \text{else} \qquad &\{S_2.next = S.next\} \\
        S_2 \qquad &\{S.code = C.code \, \| \, S_1.code \, \| \, \text{label} \| \, L_1 \, \| \, S_2.code\}
    \end{align*}

    \item 先写出SDD如下:
    
    \begin{align*}
        S \to \text{do} \, S_1 \, \text{while} \, (C) \qquad &L_1 = new(); \\
        &C.false = S.next; \\
        &C.true = L_1; \\
        &S.code = \text{label} \, \| \, L_1 \, \| \, S_1.code \, \| \, C.code
    \end{align*}

    对应SDT如下:

    \begin{align*}
        S \to \text{do} \qquad &\{L_1 = new()\} \\
        S_1 \qquad &\{C.true = L_1 \,,\ C.false = S.next\} \\
        \text{while} \, (C) \qquad &\{S.code = \text{label} \, \| \, L_1 \, \| \, S_1.code \, \| \, C.code\}
    \end{align*}

    \item 先写出SDD如下:
    
    \begin{align*}
        S \to \{\,L\,\} \qquad &L.next = S.next \,,\ S.code = L.code \\
        L \to L_1 \, S \qquad &S.next = L.next \,,\ L.code = L_1.code \, \| \, S.code \\
        L \to \varepsilon \qquad &\text{if} \, (L.next) \{L.code = \text{goto} \, \| \, L.next\} \, \text{else} \, \{L.code = \varepsilon\}
    \end{align*}

    对应SDT如下:

    \begin{align*}
        S \to \{ \qquad &\{L.next = S.next\} \\
        L \, \} \qquad &\{S.code = L.code\} \\
        L \to L_1 \qquad &\{S.next = L.next\} \\
        S \qquad &\{L.code = L_1.code \, \| \, S.code\} \\
        L \to \varepsilon \qquad &\{\text{if} \, (L.next) \, \{L.code = \text{goto} \, \| \, L.next\} \, \text{else} \, \{L.code = \varepsilon\}\}
    \end{align*}


\end{enumerate}

\end{document}